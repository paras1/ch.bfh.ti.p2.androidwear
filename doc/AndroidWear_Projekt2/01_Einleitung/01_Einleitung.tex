\chapter{Einleitung}
\label{chap:teil1_einleitung}

Dieses Dokument soll die Möglichkeiten fur eine App-Entwicklung fur eine Smartwatch {(z.B. "LG G Watch")} und Android Wear im Rahmen einer Bachelorarbeit aufzeigen.
Damit beurteilt werden kann, ob sich diese Technologie für die Projektarbeit eignet, wird zuerst der Wearable-Markt analysiert und die bereits verfügbaren Produkte und deren Features miteinander verglichen.

% Einträge im Verzeichnis erscheinen lassen ohne hier eine Referenz einzufägen
\nocite{kopka:band1}
\nocite{raichle:bibtex_programmierung}
\nocite{MiKTeX}
\nocite{KOMA}
\nocite{TeXnicCenter}
\nocite{Marti06}
\nocite{Erbsland08}
\nocite{juergens:einfuehrung}
\nocite{juergens:fortgeschritten}

\section{Was sind Wearables}
\label{sec:teil1_was_sind_wearables}

Wearables sind kompakte Computersysteme, welche während der Anwendung am Körper des Benutzers befestigt sind.
Diese kleinen Systeme sollen den Träger beim alltäglichen Leben unterstützen.

\subsection{Beispiele}
\begin{tabular}{ll}
Armbanduhren: &	Welche ständig den Puls messen und unter Kontrolle haben \\
Brillen:	& Augmented Reality, Informationen zu Umgebung mit kleinem Display \\
Fitnesstracker: & Bewegungen registrieren und auswerten \\
Hörgeräte: & Dem Träger das Hören erleichtern
\end{tabular}

\subsection{Fachbereich Informatik}
Wearables ist ein fachübergreifendes Gebiet der Informatik, einige Fachgebiete:

- Ubiquitous Computing, die Rechnerallgegenwärtig \\
- Pervasive Computing, die Vernetzung von Alltagsgegenständen \\
- Mobile Computing, mobile Mensch zu Maschinen Kommunikation \\
- M2M, Machine-to-Machine, Informationsaustausch zwischen Zielgeräten \\
- IoT, Internet of Things, dass auf den vorhergehenden Fachbereichen basiert 

\section{Zukunft}
\label{sec:teil1_zukunft}

Wearables haben die Beziehung zwischen Konsumenten und Technologie neu definiert.
Es hilft dem Träger alltägliche Probleme zu lösen ohne damit zu interagieren.

Heute werden folgenden Aktionen aktiv mit Wearables betrieben:

- Tracken von Bewegungs- und Aktivitätsdaten \\
- Pulsmessungen \\
- Notifikationen vom Smartphone empfangen \\
- Fernsteuern von Geräten \\
- Überwachen von Schlaf \\
- Personalisierter Wecker

Schon jetzt gibt es viele Anwendungen im Bereich Wearables und es werden noch viel mehr.

In der Zukunft mögliche Funktionen:

- Authentifizierung (Auto öffnen, Türen aufschliessen, usw.) \\
- Bezahlen \\
- Lokalisierung \newline
- Lifelog (Verfolgen aller Aktivitäten) \\
- Genauere evaluieren von Bewegungs- und Aktivitätsdaten \\
- e-Health \\
- uvm. 

Es verspricht interessant zu werden und garantiert weiterhin eine schnelle technologische Entwicklung.
Viele Funktionen werden dem Träger nützen ohne das dieser eine Aktion mit dem tragbarem Geräte ausführen muss.
